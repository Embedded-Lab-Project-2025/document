\documentclass[a4paper]{article}

% I made this doc by editing the template at https://www.overleaf.com/latex/templates/trinity-college-dublin-simple-slash-module-report-style/swvvktznrqbd

% Use XeLaTex or LuaLaTex to compile
\usepackage[english]{babel}
\usepackage[utf8]{inputenc}
\usepackage[T1]{fontenc}
\usepackage[a4paper,top=2cm,bottom=2cm,left=2.5cm,right=2.5cm,marginparwidth=1.75cm]{geometry}

\usepackage{fontspec}

\usepackage{listings}
\usepackage{tcolorbox}
\tcbuselibrary{listings,breakable,skins}

% Define colors for syntax highlighting
\definecolor{codebackground}{RGB}{248,248,248}
\definecolor{codeborder}{RGB}{220,220,220}
\definecolor{linenumber}{RGB}{150,150,150}
\definecolor{keywordcolor}{RGB}{0,112,192}
\definecolor{stringcolor}{RGB}{163,21,21}
\definecolor{commentcolor}{RGB}{0,128,0}
\definecolor{numbercolor}{RGB}{148,0,211}

% Configure listings package for clean syntax highlighting
\lstset{
  basicstyle=\ttfamily\small,
  keywordstyle=\color{keywordcolor}\bfseries,
  stringstyle=\color{stringcolor},
  commentstyle=\color{commentcolor}\itshape,
  numberstyle=\tiny\color{linenumber},
  numbers=left,
  numbersep=10pt,
  stepnumber=1,
  showstringspaces=false,
  breaklines=true,
  breakatwhitespace=false,
  tabsize=4,
  frame=none,
  captionpos=b,
  xleftmargin=20pt,
  framexleftmargin=20pt,
}

% Define custom code box environment
\newtcblisting{codebox}[2][]{
  listing only,
  listing options={
    language=#2,
    #1
  },
  breakable,
  enhanced,
  colback=codebackground,
  colframe=codeborder,
  arc=2pt,
  boxrule=0.5pt,
  left=5pt,
  right=5pt,
  top=5pt,
  bottom=5pt,
  before skip=10pt,
  after skip=10pt,
  overlay={
    \fill[codeborder!30] (frame.south west) rectangle ([xshift=18pt]frame.north west);
  }
}

\usepackage{amsmath}
\usepackage{graphicx}
\usepackage{tabularx}
\usepackage{xcolor}
\usepackage{float}
\usepackage[colorlinks=true,allcolors=.,urlcolor=blue]{hyperref}

\usepackage{indentfirst}
\usepackage{microtype}

\title{Final Report}
\author{Jarn Yam Group}

% \usepackage{fancyhdr}
% \pagestyle{fancy}
% \fancyhead[L]{\docTitle}
% \fancyhead[R]{\authorName}

\usepackage{svg}

% ---------- START DOC CONFIG ----------
\newcommand{\docTitle}{Final Report}
\newcommand{\docSubtitle}{Weather Station}
\newcommand{\authorName}{Weather Station}
\date{}

\usepackage{libertine}
% \setmainfont{CHULALONGKORN}
% \renewcommand{\familydefault}{\sfdefault}
\setmonofont{JetBrains Mono}[Contextuals=Alternate]
% ----------  END DOC CONFIG  ----------

\begin{document}

\begin{titlepage}
  \vspace*{20pt}

  \centering\Large

  \textbf{Final Report}

  Weather Station

  \vspace*{20pt}

  Group \\ Jarn Yam

  % \vspace*{40pt}
  \vfill

  Members

  \vspace*{20pt}

  \begin{tabular}{lll}
    Sippakorn  & Thunyahan      & 6631355721 \\
    Krissada   & Singhakachain  & 6632007521 \\
    Thanakrit  & Bunrueng       & 6632092821 \\
    Boonyakorn & Tanrattanakorn & 6632111021
  \end{tabular}

  \vfill

  Presented to \\ \textbf{Asst. Prof. Dr. Pitchaya Sitthi-amorn}

  \vfill

  %   Dec 9\textsuperscript{th}, 2025
  %   \vspace*{20pt}

  2110366 Embedded System Lab I \\
  Semester 1/2025 \\
  Department of Computer Engineering, Faculty of Engineering, \\
  Chulalongkorn University
\end{titlepage}

\tableofcontents
\newpage

\section{Objective}
The objective of this project is to design and implement a weather station system capable of monitoring and reporting various environmental parameters, including temperature, humidity, light intensity, and liquid precipitation. The system will utilize microcontroller units (MCUs) to interface with sensors, process the collected data, and transmit it to a cloud server for remote access and visualization through a web application.

\section{Equipment List}
\begin{enumerate}
  \item STM32 Nucleo F411RE Board
  \item ESP32-S3 Development Board
  \item DHT11 Temperature and Humidity Sensor
  \item Light Dependent Resistor (LDR)
  \item HW-038 Water Level Sensor
  \item SG90 Servo Motor
  \item Jumper
  \item Breadboard
  \item USB Cables
\end{enumerate}
\subsection{Setup Picture}
\begin{figure}[H]
  \centering
  \begin{minipage}[t]{0.48\textwidth}
    \centering
    \includegraphics[width=\textwidth]{assets/hardware_setup.jpg}
    \caption{Hardware Setup}
  \end{minipage}
  \hfill
  \begin{minipage}[t]{0.48\textwidth}
    \centering
    \includegraphics[width=\textwidth]{assets/rain_gauge.jpg}
    \caption{Servo Actuated Rain Gauge}
  \end{minipage}
\end{figure}

\newpage
\section{Hardware Configuration}

\subsection{Pin Connections}
\begin{figure}[H]
  \centering
  \includegraphics[width=\textwidth]{assets/stm32_ioc.png}
  \caption{STM32 Nucleo F411 Pin Configuration in STM32CubeMX}
\end{figure}

\begin{table}[H]
  \centering
  \caption{STM32 Nucleo F411 Pin Mapping and Configuration}
  \label{tab:pin_connections}
  \resizebox{\columnwidth}{!}{
    \begin{tabular}{|c|c|c|c|c|}
      \hline
      \textbf{MCU Pin} & \textbf{Board Pin} & \textbf{Pin Purpose} & \textbf{Configuration} & \textbf{Connected To} \\
      \hline
      \textbf{PA0} & \textbf{A0} & Water Level Sensor ADC Input & Analog Input & HW-038 Water Level Sensor \\
      \hline
      \textbf{PA1} & \textbf{A1} & LDR Light Sensor ADC Input & Analog Input & Light Dependent Resistor (LDR) \\
      \hline
      \textbf{PA5} & \textbf{-} & Status LED & Digital Output (Push-Pull) & On-board LED indicator \\
      \hline
      \textbf{PB0} & \textbf{A3} & DHT11 Temperature/Humidity Sensor & Digital I/O & DHT11 data pin \\
      \hline
      \textbf{PB3} & \textbf{D3} & USART1 RX (ESP32 Communication) & Alternate Function & ESP32 UART TX \\
      \hline
      \textbf{PB6} & \textbf{D10} & USART1 TX (ESP32 Communication) & Alternate Function & ESP32 UART RX \\
      \hline
  \end{tabular}}
\end{table}

\begin{table}[H]
  \centering
  \caption{ESP32-S3 Pin Mapping and Configuration}
  \label{tab:esp32_pin_connections}
  \resizebox{\columnwidth}{!}{
    \begin{tabular}{|c|c|c|c|}
      \hline
      \textbf{GPIO Pin} & \textbf{Pin Purpose} & \textbf{Configuration} & \textbf{Connected To} \\
      \hline
      \textbf{GPIO 16} & UART RX (STM32 Communication) & Serial Input & STM32 UART TX \\
      \hline
      \textbf{GPIO 17} & UART TX (STM32 Communication) & Serial Output & STM32 UART RX\\
      \hline
      \textbf{GPIO 14} & Servo Motor Control & PWM Output & Rain Gauge Servo Motor \\
      \hline
      \textbf{GPIO 48} & RGB Status LED & NeoPixel Data & WS2812B RGB LED \\
      \hline
  \end{tabular}}
\end{table}

\subsection{Additional Configuration Details}

\subsubsection{ADC Configuration}
\begin{itemize}
  \item \textbf{ADC1} is used for both analog sensors (LDR and Water Level)
  \item Resolution: 12-bit (0-4095 range)
  \item Channels dynamically configured in software
  \item LDR uses ADC Channel 1, Water sensor uses ADC Channel 0
\end{itemize}

\subsubsection{UART Configuration}
\begin{itemize}
  \item \textbf{USART1}: 115200 baud, 8N1, TX/RX mode - Primary communication with ESP32
  \item \textbf{USART2}: 115200 baud, 8N1, TX/RX mode - Debug output via ST-Link VCP
\end{itemize}

\subsubsection{Timer Configuration}
\begin{itemize}
  \item \textbf{TIM1} is configured for microsecond timing delays required by DHT11 sensor protocol
\end{itemize}

\subsubsection{Power Management}
\begin{itemize}
  \item Most unused pins are configured as analog inputs (high impedance) to minimize power consumption
  \item ADC peripherals are started/stopped dynamically to conserve power during sensor readings
\end{itemize}

\newpage
\section{System Architecture}
\subsection{Weather Station Subsystem}
The Weather Station Subsytem is responsible for collecting weather data from the environment, namely temperature,
humidity, light intensity, and liquid precipitation.

The subsystem utilizes STM32 Nucleo F411 board as the primary microcontroller unit (MCU) to interface with the sensors.
The STM32 then processes the raw data from the sensors and transmits the processed data to ESP32S3 via UART communication protocol.
Finally, the ESP32S3 sends the data to the cloud server through MQTT to be displayed on the web application.

\subsubsection{Sensors}
\begin{table}[H]
  \centering
  \caption{Weather Station Sensors}
  \label{tab:sensor_summary}
  \resizebox{\columnwidth}{!}{
    \begin{tabular}{|c|c|c|c|}
      \hline
      \textbf{Sensor} & \textbf{Type} & \textbf{Interface} & \textbf{Measured Data} \\
      \hline
      DHT22 & Temperature/Humidity & One-wire digital & Temperature (\textdegree C), Humidity (\%) \\
      \hline
      LDR & Light Intensity & Analog voltage & Light Intensity (\%) \\
      \hline
      Water Level Sensor & Liquid Precipitation & Analog voltage & Water Level (mm) \\
      \hline
  \end{tabular}}
\end{table}
\subsubsection{STM32 to ESP32S3 Communication}
The STM32 microcontroller communicates with the ESP32S3 using UART protocol.
The ESP32S3 then sends a char `R' to request data from the STM32.
Upon receiving the request, the STM32 respond with the latest data packet.
The data packet is sent as a raw byte stream with the following structure:
\begin{table}[H]
  \centering
  \caption{UART Data Packet Structure}
  \label{tab:uart_packet_structure}
  \begin{tabular}{|c|c|c|}
    \hline
    \textbf{Field} & \textbf{Type} & \textbf{Size (bytes)} \\
    \hline
    Start Byte & uint8\_t & 1 \\
    \hline
    Temperature & uint8\_t & 1 \\
    \hline
    Humidity & uint8\_t & 1 \\
    \hline
    Light Intensity & float & 4 \\
    \hline
    Water Level & float & 4 \\
    \hline
    Checksum & uint8\_t & 1 \\
    \hline
  \end{tabular}
\end{table}
\subsubsection{MQTT Communication}
The ESP32S3 connects to a Wi-Fi network and establishes a connection to the MQTT broker.
It will periodically publish the weather data received from the STM32 to the ``sensor/data'' topic in CSV format.
The ESP32S3 also subscribes to the ``sensor/action'' topic and which will allows for the reset of the rain gauge when ``resetRainGauge'' command is received.
\subsubsection{Additional Feature}
In addition to MQTT communication, the ESP32S3 also controls a servo motor with PWM signal to reset the rain gauge and an RGB LED to indicate system status.
\begin{table}[H]
  \centering
  \caption{ESP32S3 Status LED Color Mapping}
  \label{tab:status_color}
  \begin{tabular}{|c|c|c|c|}
    \hline
    \textbf{Status} & \textbf{Color} & \textbf{RGB Value} & \textbf{Meaning} \\
    \hline
    Normal Operation & Green & (0, 255, 0) & All systems OK \\
    \hline
    Wi-Fi Not Connected & Red & (255, 0, 0) & Wi-Fi connection lost or unavailable \\
    \hline
    MQTT Not Connected & Blue & (0, 0, 255) & MQTT broker not connected \\
    \hline
    Sensor Error & Yellow & (255, 255, 0) & Sensor communication or data error \\
    \hline
  \end{tabular}
\end{table}

\subsection{Data Collection Subsystem}
\label{sub:data_collection_sub}
The Data Collection Subsystem serves as the central hub for receiving, processing, storing, and distributing weather data from the ESP32S3 device to the web application. This subsystem consists of an MQTT broker, a backend server, and a database system.

\subsubsection{Data Flow Diagram}
\begin{figure}[H]
  \centering
  \includegraphics[width=\textwidth]{assets/data_flow_diagram.pdf}
  \caption{Diagram depicting flow of data between each major components}
\end{figure}

\subsubsection{MQTT Broker}
The system utilizes an MQTT broker as the primary communication protocol between the ESP32S3 and the website. MQTT was selected for its lightweight nature, low bandwidth requirements, and publish-subscribe messaging pattern, which is ideal for IoT applications.

Due to limitation on the browser's ability to directly connect via MQTT protocol, Eclipse Mosquitto was chosen for its ability to open a WebSocket interface, allowing the web application to communicate with the broker directly.

\begin{itemize}
  \item \textbf{Broker}: Eclipse Mosquitto
  \item \textbf{Protocol}: MQTT and WebSocket
  \item \textbf{Topics}:
    \begin{itemize}
      \item \texttt{sensor/data}: Published by ESP32S3 containing weather readings in CSV format
      \item \texttt{sensor/action}: Subscribed by ESP32S3 to receive \texttt{resetRainGauge} command
    \end{itemize}
\end{itemize}

\subsubsection{Backend Server}
The backend server subscribes to the MQTT broker to receive sensor data, stores it in the database, and provides historical data for the web application.
It has a single API endpoint which is \texttt{/api/history}.

\begin{itemize}
  \item \textbf{Technology Stack}: Go with Fiber framework
  \item \textbf{MQTT Client Library}: Eclipse Paho
    % \item \textbf{Key Functions}:
    %   \begin{itemize}
    %     \item Subscribe to \texttt{sensor/data} topic and parse incoming CSV data
    %     \item Validate and sanitize sensor readings
    %     \item Store data with timestamps in the database
    %     \item Publish commands to \texttt{sensor/action} topic based on web requests
    %     \item Expose RESTful API endpoints for data retrieval and device control
    %   \end{itemize}
\end{itemize}

\subsubsection{API Response Format}

The \texttt{/api/history} endpoint returns a JSON object containing an array of historical sensor readings:

\begin{codebox}{python}
  {
    "data": [
      {
        "id": 123,
        "timestamp": "2025-12-09T10:30:00Z",
        "temperature": 25.5,
        "humidity": 60.2,
        "light": 75.8,
        "rain": 12.3
      },
      ...
    ],
    "count": 100
  }
\end{codebox}

\begin{table}[H]
  \centering
  \caption{API Response Fields}
  \label{tab:api_response_fields}
  \begin{tabular}{|l|l|l|}
    \hline
    \textbf{Field} & \textbf{Type} & \textbf{Description} \\
    \hline
    data & Array & Array of historical data points \\
    \hline
    count & Integer & Total number of records returned \\
    \hline
    id & Integer & Unique identifier for each reading \\
    \hline
    timestamp & String (ISO 8601) & Time when data was recorded \\
    \hline
    temperature & Float & Temperature in Celsius degrees \\
    \hline
    humidity & Float & Relative humidity in percentage \\
    \hline
    light & Float & Light intensity in percentage \\
    \hline
    rain & Float & Precipitation level in millimeters \\
    \hline
  \end{tabular}
\end{table}

\subsubsection{Database}
The database stores historical weather data for visualization and analysis.

\begin{itemize}
  \item \textbf{Database System}: PostgreSQL 17
  \item \textbf{Data Schema}:
    \begin{table}[H]
      \centering
      \caption{Database Schema for Weather Data}
      \label{tab:database_schema}
      \begin{tabular}{|l|l|}
        \hline
        \textbf{Field} & \textbf{Data Type} \\
        \hline
        id & SERIAL PRIMARY KEY \\
        \hline
        timestamp & TIMESTAMP \\
        \hline
        temperature & DOUBLE PRECISION \\
        \hline
        humidity & DOUBLE PRECISION \\
        \hline
        light & DOUBLE PRECISION \\
        \hline
        rain & DOUBLE PRECISION \\
        \hline
      \end{tabular}
    \end{table}
\end{itemize}

% \subsubsection{Data Flow}
% \begin{enumerate}
%   \item ESP32S3 publishes sensor data to \texttt{sensor/data} topic in CSV format
%   \item Backend server receives the message via MQTT subscription
%   \item Server parses and validates the data
%   \item Validated data is stored in the database with a timestamp
%   \item Web application requests data via RESTful API
%   \item Server queries database and returns JSON response
%   \item User actions from web application are published to \texttt{sensor/action} topic
%   \item ESP32S3 receives commands and executes corresponding actions
% \end{enumerate}

\subsection{Website Subsystem}
\label{sub:website_sub}
The Website Subsystem provides a user-friendly interface for monitoring real-time weather data, viewing historical trends, and controlling device functions.
% The web application is designed to be responsive and accessible from various devices including desktops, tablets, and smartphones.
SvelteKit is selected for its lightweight nature and ease of use.

\subsubsection{Frontend Architecture}
\begin{itemize}
  \item \textbf{Framework}: SvelteKit and Tailwind CSS
  \item \textbf{State Management}: Svelte Store for application state
  \item \textbf{Data Visualization}: D3.js for graphs and Three.js for 3D elements
\end{itemize}

\subsubsection{Pages and Features}
\begin{enumerate}
  \item \textbf{Real-time Dashboard}
    \begin{itemize}
      \item Stunning 3D house scene with dynamic lighting and rain animation
      \item Display current temperature, humidity, light intensity, and precipitation levels
      \item ``DRAIN'' button for manually issuing the board to drain the rain gauge
      \item Updates in real-time as new data arrives
    \end{itemize}

    \begin{figure}[H]
      \centering
      \includegraphics[width=\textwidth]{assets/home_rain.png}
      \caption{Screenshot of Real-time Dashboard Page}
    \end{figure}

  \item \textbf{Historical Data Visualization}
    \begin{itemize}
      \item Line charts showing collected sensor data over time
      \item Date range selector for custom time periods
      \item Fetches historical data from backend API
      \item Updates in real-time as new data arrives
    \end{itemize}

    \begin{figure}[H]
      \centering
      \begin{minipage}{0.48\textwidth}
        \centering
        \includegraphics[width=\textwidth]{assets/charts_1.png}
      \end{minipage}
      \hfill
      \begin{minipage}{0.48\textwidth}
        \centering
        \includegraphics[width=\textwidth]{assets/charts_2.png}
      \end{minipage}
      \caption{Screenshot of Historical Data Visualization Page}
    \end{figure}
\end{enumerate}

\subsubsection{Deployment}
Both \nameref{sub:website_sub} and \nameref{sub:data_collection_sub} subsystems are deployed on a home server using Docker containers for portability.

\newpage

\section{Role and Responsibility}
\begin{enumerate}
  \item Sippakorn Thunyahan (Project Manager):
    Overall project coordination, document review, project setup
  \item Krissada Singhakachain (Backend Engineer): Server setup, web application development
  \item Thanakrit Bunrueng (System Architect): Design system architecture, design communication interface
  \item Boonyakorn Tanrattanakorn (Embedded System Programmer): Implement sensor integration, embedded firmware development
\end{enumerate}

\section{Project Timeline}
\textbf{Duration:} November 10 -- December 8, 2025 (4 weeks)

\subsection{Week 1 (Nov 10-16)}
\begin{itemize}
  \item \textbf{Sippakorn:} Project kickoff, repository setup
  \item \textbf{Krissada:} Research MQTT brokers, setup dev environment
  \item \textbf{Thanakrit:} Design system architecture, define protocols
  \item \textbf{Boonyakorn:} Study STM32 HAL, setup IDE, test basic I/O
\end{itemize}

\subsection{Week 2 (Nov 17-23)}
\begin{itemize}
  \item \textbf{Sippakorn:} Review architecture, track milestones
  \item \textbf{Krissada:} Deploy MQTT broker, build web app, create APIs
  \item \textbf{Thanakrit:} Define MQTT topics, design data packets
  \item \textbf{Boonyakorn:} Implement DHT11 driver, configure ADC, develop UART
\end{itemize}

\subsection{Week 3 (Nov 24-30)}
\begin{itemize}
  \item \textbf{Sippakorn:} Integration testing, update documentation
  \item \textbf{Krissada:} Integrate MQTT client, add data logging, build UI
  \item \textbf{Thanakrit:} Validate protocols
  \item \textbf{Boonyakorn:} Integrate sensors, implement ESP32 firmware, add servo/LED
\end{itemize}

\subsection{Week 4 (Dec 1-8)}
\begin{itemize}
  \item \textbf{Sippakorn:} Final testing, check results, prepare presentation
  \item \textbf{Krissada:} Deploy production
  \item \textbf{Thanakrit:} Performance validation, finalize documentation
  \item \textbf{Boonyakorn:} Bug fixes, power optimization
\end{itemize}

\subsection{Key Milestones}
\begin{itemize}
  \item \textbf{Nov 16:} Architecture approved, dev environment ready
  \item \textbf{Nov 23:} Individual subsystems functional
  \item \textbf{Nov 30:} Full system integration complete
  \item \textbf{Dec 8:} Final testing and delivery
\end{itemize}

\section{Appendix}
\begin{itemize}
  \item  Source code repository: \url{https://github.com/orgs/Embedded-Lab-Project-2025/repositories}
\end{itemize}

\end{document}