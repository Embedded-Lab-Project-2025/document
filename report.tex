\documentclass[a4paper]{article}

% I made this doc by editing the template at https://www.overleaf.com/latex/templates/trinity-college-dublin-simple-slash-module-report-style/swvvktznrqbd

% Use XeLaTex or LuaLaTex to compile
\usepackage[english]{babel}
\usepackage[utf8]{inputenc}
\usepackage[T1]{fontenc}
\usepackage[a4paper,top=2cm,bottom=2cm,left=2.5cm,right=2.5cm,marginparwidth=1.75cm]{geometry}

\usepackage{fontspec}

\usepackage{tcolorbox}
\tcbuselibrary{minted,breakable,xparse,skins}

% Snippet from https://tex.stackexchange.com/questions/475826/how-to-present-a-python-code-snippet-efficiently-in-latex
\definecolor{bg}{gray}{0.95}
\DeclareTCBListing{mintedbox}{O{}m!O{}}{%
  breakable=true,
  listing engine=minted,  
  listing only,
  minted language=#2,
  minted style=default,
  minted options={%
    linenos,
    gobble=0,
    breaklines=true,
    breakafter=,,
    fontsize=\small,
    numbersep=8pt,
  #1},
  boxsep=0pt,
  left skip=0pt,
  right skip=0pt,
  left=25pt,
  right=0pt,
  top=3pt,
  bottom=3pt,
  before skip=12pt,
  after skip=12pt,
  % arc=5pt,
  leftrule=0pt,
  rightrule=0pt,
  bottomrule=0pt,
  toprule=0pt,
  % colback=bg,
  % colframe=orange!70,
  enhanced,
  overlay={%
    \begin{tcbclipinterior}
      \fill[gray!20!white] (frame.south west) rectangle ([xshift=20pt]frame.north west);
  \end{tcbclipinterior}},
#3}

% font for line number
\renewcommand{\theFancyVerbLine}{%
  \textcolor{gray}{\scriptsize\arabic{FancyVerbLine}}%
}

\usepackage{amsmath}
\usepackage{graphicx}
\usepackage{tabularx}
\usepackage{xcolor}
\usepackage[colorlinks=true,allcolors=.,urlcolor=blue]{hyperref}

\usepackage{indentfirst}
\usepackage{microtype}

\title{Final Report}
\author{Jarn Yam Group}

% \usepackage{fancyhdr}
% \pagestyle{fancy}
% \fancyhead[L]{\docTitle}
% \fancyhead[R]{\authorName}

\usepackage{svg}

% ---------- START DOC CONFIG ----------
\newcommand{\docTitle}{Final Report}
\newcommand{\docSubtitle}{Weather Station}
\newcommand{\authorName}{Weather Station}
\date{}

\usepackage{libertine}
% \setmainfont{CHULALONGKORN}
% \renewcommand{\familydefault}{\sfdefault}
% \setmonofont{JetBrains Mono}[Contextuals=Alternate]
% ----------  END DOC CONFIG  ----------

\begin{document}

\begin{titlepage}
  \vspace*{20pt}

  \centering\Large

  \textbf{Final Report}

  Weather Station

  \vspace*{20pt}

  Group \\ Jarn Yam

  % \vspace*{40pt}
  \vfill

  Members

  \vspace*{20pt}

  \begin{tabular}{lll}
    Sippakorn  & Thunyahan      & 6631355721 \\
    Krissada   & Singhakachain  & 6632007521 \\
    Thanakrit  & Bunrueng       & 6632092821 \\
    Boonyakorn & Tanrattanakorn & 6632111021
  \end{tabular}

  \vfill

  Presented to \\ \textbf{Asst. Prof. Dr. Pitchaya Sitthi-amorn}

  \vfill

  %   Dec 9\textsuperscript{th}, 2025
  %   \vspace*{20pt}

  2110366 Embedded System Lab I \\
  Semester 1/2025 \\
  Department of Computer Engineering, Faculty of Engineering, \\
  Chulalongkorn University
\end{titlepage}

\tableofcontents
\newpage

\section{Objective}
The objective of this project is to design and implement a weather station system capable of monitoring and reporting various environmental parameters, including temperature, humidity, light intensity, and liquid precipitation. The system will utilize microcontroller units (MCUs) to interface with sensors, process the collected data, and transmit it to a cloud server for remote access and visualization through a web application.

\section{Equipment List}
\begin{enumerate}
  \item STM32 Nucleo F411RE Board
  \item ESP32-S3 Development Board
  \item DHT11 Temperature and Humidity Sensor
  \item Light Dependent Resistor (LDR)
  \item HW-038 Water Level Sensor
  \item SG90 Servo Motor
  \item Jumper
  \item Breadboard
  \item USB Cables
\end{enumerate}


\newpage
\section{Hardware Configuration}

\begin{table}[H]
\centering
\caption{STM32 Nucleo F411 Pin Mapping and Configuration}
\label{tab:pin_connections}
\resizebox{\columnwidth}{!}{\begin{tabular}{|c|c|c|c|c|}
\hline
\textbf{MCU Pin} & \textbf{Board Pin} & \textbf{Pin Purpose} & \textbf{Configuration} & \textbf{Connected To} \\
\hline
\textbf{PA0} & \textbf{A0} & Water Level Sensor ADC Input & Analog Input & HW-038 Water Level Sensor \\
\hline
\textbf{PA1} & \textbf{A1} & LDR Light Sensor ADC Input & Analog Input & Light Dependent Resistor (LDR) \\
\hline
\textbf{PA5} & \textbf{D13} & Status LED & Digital Output (Push-Pull) & On-board LED indicator \\
\hline
\textbf{PA9} & \textbf{D8} & USART1 TX (ESP32 Communication) & Alternate Function & ESP32 UART RX \\
\hline
\textbf{PA10} & \textbf{D2} & USART1 RX (ESP32 Communication) & Alternate Function & ESP32 UART TX \\
\hline
\textbf{PB0} & \textbf{D3} & DHT11 Temperature/Humidity Sensor & Digital I/O & DHT11 data pin \\
\hline
\textbf{PC13} & \textbf{B1} & User Button & Digital Input (Interrupt) & On-board user button \\
\hline
\end{tabular}}
\end{table}

\begin{table}[H]
\centering
\caption{ESP32-S3 Pin Mapping and Configuration}
\label{tab:esp32_pin_connections}
\resizebox{\columnwidth}{!}{\begin{tabular}{|c|c|c|c|} 
\hline
\textbf{GPIO Pin} & \textbf{Pin Purpose} & \textbf{Configuration} & \textbf{Connected To} \\
\hline
\textbf{GPIO 16} & UART2 RX (STM32 Communication) & Serial Input & STM32 UART TX (PA9) \\
\hline
\textbf{GPIO 17} & UART2 TX (STM32 Communication) & Serial Output & STM32 UART RX (PA10) \\
\hline
\textbf{GPIO 14} & Servo Motor Control & PWM Output & Rain Gauge Servo Motor \\
\hline
\textbf{GPIO 48} & RGB Status LED & NeoPixel Data & WS2812B RGB LED \\
\hline
\end{tabular}}
\end{table}

\subsection{Additional Configuration Details}

\subsubsection{ADC Configuration}
\begin{itemize}
    \item \textbf{ADC1} is used for both analog sensors (LDR and Water Level)
    \item Resolution: 12-bit (0-4095 range)
    \item Channels dynamically configured in software
    \item LDR uses ADC Channel 1, Water sensor uses ADC Channel 0
\end{itemize}

\subsubsection{UART Configuration}
\begin{itemize}
    \item \textbf{USART1}: 115200 baud, 8N1, TX/RX mode - Primary communication with ESP32
    \item \textbf{USART2}: 115200 baud, 8N1, TX/RX mode - Debug output via ST-Link VCP
\end{itemize}

\subsubsection{Timer Configuration}
\begin{itemize}
    \item \textbf{TIM1} is configured for microsecond timing delays required by DHT11 sensor protocol
\end{itemize}

\subsubsection{Power Management}
\begin{itemize}
    \item Most unused pins are configured as analog inputs (high impedance) to minimize power consumption
    \item ADC peripherals are started/stopped dynamically to conserve power during sensor readings
\end{itemize}

\newpage
\section{System Architecture}
\subsection{Weather Station Subsystem}
  The Weather Station Subsytem is responsible for collecting weather data from the environment, namely temperature, 
  humidity, light intensity, and liquid precipitation. 

  The subsystem utilizes STM32 nucleo F411 board as the primary microcontroller unit (MCU) to interface with the sensors.
  The STM32 then processes the raw data from the sensors and transmits the processed data to ESP32S3 via UART communication protocol.
  Finally, the ESP32S3 sends the data to the cloud server through MQTT to be displayed on the web application.

  \subsubsection{Sensors}
    \begin{table}[H]
    \centering
    \caption{Weather Station Sensors}
    \label{tab:sensor_summary}
    \resizebox{\columnwidth}{!}{\begin{tabular}{|c|c|c|c|}
      \hline
      	\textbf{Sensor} & \textbf{Type} & \textbf{Interface} & \textbf{Measured Data} \\
      \hline
      DHT22 & Temperature/Humidity & One-wire digital & Temperature (\textdegree C), Humidity (\%) \\
      \hline
      LDR & Light Intensity & Analog voltage & Light Intensity (\%) \\
      \hline
      Water Level Sensor & Liquid Precipitation & Analog voltage & Water Level (mm) \\
      \hline
    \end{tabular}}
    \end{table}
\subsubsection{STM32 to ESP32S3 Communication}
  The STM32 microcontroller communicates with the ESP32S3 using UART protocol.
  The ESP32S3 then sends a char 'R' to request data from the STM32.
  Upon receiving the request, the STM32 respond with the latest data packet.
  The data packet is sent as a raw byte stream with the following structure:
  \begin{table}[H]
    \centering
    \caption{UART Data Packet Structure}
    \label{tab:uart_packet_structure}
    \begin{tabular}{|c|c|c|}
      \hline
      	\textbf{Field} & \textbf{Type} & \textbf{Size (bytes)} \\
      \hline
      Start Byte & uint8\_t & 1 \\
      \hline
      Temperature & uint8\_t & 1 \\
      \hline
      Humidity & uint8\_t & 1 \\
      \hline
      Light Intensity & float & 4 \\
      \hline
      Water Level & float & 4 \\
      \hline
      Checksum & uint8\_t & 1 \\
      \hline
    \end{tabular}
  \end{table}
\subsection{ESP32S3 to MQTT Backend Communication}
The ESP32S3 connects to a Wi-Fi network and establishes a connection to the MQTT broker.
It will periodically publishes the weather data received from the STM32 to the "sensor/data" topic in CSV format.
The ESP32S3 also subscribes to the "sensor/action" topic and which will allows for the reset of the rain gauge when "resetRainGauge" command is received.

\newpage
\section{Source Code}

\begin{mintedbox}{python}
language = 'Python'
print(f'This is {language}')
s = f'This is how to add code snippet.'
print(s)
\end{mintedbox}

\subsection{Diagram}
% TODO: change to a more professional diagram
\begin{figure}[H]
  \centering
  \includegraphics[width=\textwidth]{assets/system_overview_diagram.png}
  \caption{System overview diagram}
\end{figure}

\section{Role and Responsibility}
\begin{enumerate}
  \item Sippakorn Thunyahan (Project Manager): 
  Overall project coordination, document review, project setup
  \item Krissada Singhakachain (Backend Engineer): Cloud server setup, Web application development
  \item Thanakrit Bunrueng (System Architet): Design system architecture, Design communication interface
  \item Boonyakorn Tanrattanakorn (Embedded System Programmer): Implement sensor integration, embedded firmware development

\section{Appendix}
\begin{itemize}
  \item  Source code repository: \url{https://github.com/Embedded-Lab-Project-2025}
\end{itemize}
\end{enumerate}

\end{document}